
\documentclass{standalone}
\usepackage[american,oldvoltagedirection,siunitx]{circuitikz}
\usepackage{tikz}
\usetikzlibrary{calc,arrows}
\usetikzlibrary{shapes.geometric}




\begin{document}



\begin{circuitikz}[] \draw

(0,0) node [npn,xscale=1,rotate=0,anchor=west,xshift=-9.75cm] (npn1) {$BC548$}

(npn1.collector) ++ (2,0) node [npn,xscale=1,rotate=0] (npn2) {$BC548$}

(npn1.collector) to[R,l=$1\,K\Omega$] (npn2.base)

(npn1.base) ++ (-2,0) to[R,l=$1\,K\Omega$] (npn1.base)

(npn1.base) ++ (-2,0) node[ocirc]{$V_{in}$}

(npn2.collector) ++ (2,0) node[ocirc]{$V_{out}$}
(npn2.collector) ++ (0,2) to[R,l=$10\,K\Omega$] (npn2.collector)
(npn1.collector) ++ (0,2) to[R,l=$10\,K\Omega$] (npn1.collector)

(npn2.collector) -- ++ (2,0) 




(npn1.emitter) -- ++ (0,-1) node[ground]{}
(npn2.emitter) -- ++ (0,-1) node[ground]{}


%(xnor.in 1) ++ (-1,0) node (I)     [anchor=east,xshift=-.75cm]           {$I$}

;\end{circuitikz}


 
\end{document}