\input{Config.tex}

%\documentclass{standalone}
%\usepackage[american,oldvoltagedirection,siunitx]{circuitikz}
%\usepackage{tikz}
\usetikzlibrary{calc,arrows}
\usetikzlibrary{shapes.geometric}

\tikzset{
  multiplexer/.style={
    draw,
    trapezium,
    shape border uses incircle, 
    shape border rotate=270,
    minimum size=18pt
  }  
}




\begin{document}

\def\DFF(#1)#2#3#4{%
  \begin{scope}[shift={(#1)}]
    \draw (0,0) rectangle (2,3);
    \node at (1.75,2.5) {$Q_#4$};
    \node at (1.75,0.5) {$\bar{Q_#4}$};
    \node at (0.25,2.5) {$D$};
    \node at (1,1.5) {#3};
	%\node at (1,1.5) {\scriptsize $CLK$};
    \draw (2,2.5) -- ++(0.25,0) coordinate (#2 Q);
    
    \draw (2,0.5) -- ++(0.25,0) coordinate (#2 Q!);
    
    \draw (0,2.5) -- ++(-0.25,0) coordinate (#2 D);
    
    \draw (0,1.5) -- ++(-0.25,0) coordinate (#2 CLK);
	\draw (0,1.75) -- (0.25,1.5) -- (0,1.25);
	
	
	
  \end{scope}
}

\begin{circuitikz}[every path/.style={},>=triangle 45] 

  \DFF(0,0){a}{$D_0$}{0}
  \DFF(0,4){b}{$D_1$}{1}
  
  
\draw
(a D) ++ (-1,0) node (and1) [american and port,number inputs=3]{}

(b D) ++ (-1,0) node (and2) [american and port,number inputs=2]{}

(and2.in 2) ++ (-1,-1) node (or1) [american or port,number inputs=2]{}

%nodes
(and2.in 1) ++ (-3,0) node (X)     [anchor=east,xshift=-.75cm]           {$X$}






%%connections

(and1.out) -- (a D)
(and2.out) -- (b D)
(X) -- (and2.in 1)
(or1.out) |- (and2.in 2)
(X) ++ (0.5,0) |- (and1.in 2)
(a Q) -- ++ (0.5,0)
(b Q) -- ++ (0.5,0)
(a Q!) -- ++ (0.5,0)
(b Q!) -- ++ (0.5,0)
(a Q) ++ (0.5,0) -- ++ (0,1) -| (or1.in 2)
(b Q) ++ (0.5,0) -- ++ (0,1) -| (or1.in 1)

(b Q!) ++ (0.5,0) -- ++ (0,-0.75) -| (and1.in 1)
(a Q!) ++ (0.5,0) -- ++ (0,-0.75) -| (and1.in 3)




(a CLK) ++ (-3,-1) node (CLK) [anchor=east,xshift=-.2cm] {$CLK$}

(a CLK)  -- (b CLK)
(a CLK) |- ++ (-3,-1)






;\end{circuitikz}


 
\end{document}